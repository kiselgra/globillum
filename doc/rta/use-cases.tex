\documentclass[a4paper,11pt]{scrartcl}

% packages and settings [[[

% input	[[[
\usepackage[utf8]{inputenc}
\usepackage{ngerman}
% ]]]
% math/symbols [[[

\usepackage{latexsym}
\usepackage{amsfonts}	
\usepackage{amsmath}
\usepackage{amssymb}

% \usepackage{wasysym}	\Smile
% ]]]
% tables/graphics/algorithms/listings [[[

\usepackage{booktabs}
\usepackage{multirow}
\usepackage{graphicx}
\usepackage{listings}
\usepackage{varwidth}
% \usepackage{xcolor}
\usepackage[usenames,dvipsnames,svgnames,table]{xcolor}

\usepackage{algorithmicx}
\usepackage{algorithm} % http://ctan.org/pkg/algorithms
\usepackage[noend]{algpseudocode} % http://ctan.org/pkg/algorithmicx
\newcommand*{\Let}[2]{\State #1 $\gets$ \parbox[t]{\linegoal}{#2\strut}}
% \algnewcommand\algorithmicinput{\textbf{INPUT: }}
% \algnewcommand\Input{\item[\algorithmicinput]}
% \algnewcommand\algorithmicoutput{\textbf{OUTPUT: }}
% \algnewcommand\Output{\item[\algorithmicoutput]}

% \usepackage{algorithmic}
% \usepackage[boxed]{algorithm}
\usepackage{subfigure}
\usepackage[margin=10pt,font=small,labelfont=bf]{caption}		
	% \caption zb. in \figure wird \small

% nicer comment formatting in algorithmic env.
% \renewcommand{\algorithmiccomment}[1]{\hspace{\stretch{1}}// #1$\quad$}
% \floatname{algorithm}{Algorithmus}

% drawin trees
% \usepackage{pstricks,pst-node,pst-tree}

% ]]]
% paragraph formatting [[[
\usepackage{a4wide}

\usepackage{parskip}
\parindent0pt
\setlength{\parskip}{1ex plus 0.5ex minus 0.2 ex}

% ]]]

% ]]]
% my commands [[[

% my math commands [[[

\newcommand{\R}{\mathbb{R}}
\newcommand{\N}{\mathbb{N}}
\newcommand{\mts}{\;|\;}
\newcommand{\eqs}{\;=\;}
\newcommand{\D}{\mathrm{d}}
\newcommand{\Z}{\mathbb{Z}}
\renewcommand{\P}{\mathbb{P}}
\newcommand{\area}{\mathrm{area}}
\newcommand{\matrixtwo}[2]{\left(\begin{array}{cc}#1\\#2\end{array}\right)}
\newcommand{\matrixthree}[3]{\left(\begin{array}{ccc}#1\\#2\\#3\end{array}\right)}
\newcommand{\matrixfour}[4]{\left(\begin{array}{cccc}#1\\#2\\#3\\#4\end{array}\right)}
\newcommand{\vect}[3]{\left(\begin{array}{c}#1\\#2\\#3\end{array}\right)}
\newcommand{\vectfour}[4]{\left(\begin{array}{c}#1\\#2\\#3\\#4\end{array}\right)}
\newcommand{\norm}{\mathrm{norm}}
\newcommand{\scalPr}[2]{\left\langle #1|#2 \right\rangle}

% ]]]
% definitions [[[

\newcounter{currdefinition}
\newcommand{\defnr}[1]{\arabic{#1}}
\newcommand{\defref}[1]{Definition \defnr{#1}}
\newcommand{\definition}[2]
	{	\addtocounter{currdefinition}{1}%
		\newcounter{#1}%
		\setcounter{#1}{\thecurrdefinition}%
		\label{#1}%
		\paragraph{\textbf{Definition \thecurrdefinition:}}\ \\#2}

% ]]]
% misc commands [[[

\newcommand{\ra}{$\rightarrow\ $}
\newcommand{\wichtig}[1]{\framebox{\begin{minipage}[h]{\textwidth}#1\end{minipage}} }
\newcommand{\nachvollz}[1]{\paragraph{\textbf{Nachvollziehen:}}\ \\#1}
\newcommand{\tableitemization}{\hspace{10pt}$\bullet$&}
\newcommand{\HELP}[1]{\ \\\ \\\Huge Hilfe! \normalsize\\$\rightarrow$\ #1\ \\\ \\}

\newcommand{\inlinecode}[1]{{\tt #1}}
\newcommand{\inlineshell}[1]{{\tt #1}}

% ]]]
% levels [[[

\newcommand{\levelA}[1]{\section{#1}}
\newcommand{\levelB}[1]{\subsection{#1}}
\newcommand{\levelC}[1]{\subsubsection{#1}}
\newcommand{\levelD}[1]{\paragraph{#1:}\ \\}
\newcommand{\levelDn}[1]{\paragraph{#1:}}    % level D ohne "\ \\" --> bei paragraph oder itemization direkt danach sonst zu großer zwischenraum
\newcommand{\levelE}[1]{\paragraph{#1:}\ \\}
\newcommand{\levelEn}[1]{\paragraph{#1:}}    % level D ohne "\ \\" --> bei paragraph oder itemization direkt danach sonst zu großer zwischenraum

% ]]]

% ]]]

\usepackage{comment}

\definecolor{numbergray}{gray}{0.5}

% settings for listing environment
\lstset{language=C++,
		basicstyle=\ttfamily\footnotesize,
		frame=lines,
        breaklines=true,
        breakatwhitespace=true,
		keywordstyle=\color{OliveGreen},
		numbers=left,
		numberstyle=\color{numbergray}\scriptsize,
		framexleftmargin=5mm,
		xleftmargin=5mm,
		tabsize=4,
		commentstyle=\color{CadetBlue},
		stringstyle=\color{Bittersweet},
		showstringspaces=false,
		escapechar=@,
		% C++ KEYWORDS
		morekeywords={
			float3,float2,vec3f,vec2f,
			std,string,vector},
		classoffset=1,
		% RTA KEYWORDS
		morekeywords={
			rta,cgls,cuda,simple_triangle,simple_aabb,convert_scene_to_cuda_triangle_data,basic_flat_triangle_list,cpu_ftl,
			connection,rt_set,plugin_create_rt_set,box_t,tri_t,acceleration_structure,
			basic_ctor,basic_rt,basic_as,build,basic_acceleration_structure,as,rt,
			input_flat_triangle_list,expects_host_triangles,cuda_triangle_data,
			declare_traits_types,cam_ray_generator_shirley,gpu_ray_bouncer,tri_ptr,trace_further-bounces,bounce,identification,
			trace_further_bounces,gpu_last_intersection,gpu_origin,gpu_direction,new_pass,valid,
			raytracer,basic_raytracer,bouncer},
		keywordstyle=\color{Maroon},
		classoffset=0
}


\title{RTA Use Cases}
\subtitle{Examples of how we put RTA to use \\to implement certain rendering techniques}
\author{Kai}
% \date{\today}

% \setlength{\unitlength}{1mm}
\newenvironment{alg}{
	\medskip\hrule
	\begin{algorithmic}}
	{\end{algorithmic}
	\hrule\medskip}

% 	\noindent\fbox{
% 	\begin{varwidth}{\dimexpr\linewidth-2\fboxsep-2\fboxrule\relax}
%	...}

\begin{document}

\maketitle\thispagestyle{empty}
% \tableofcontents

\section{Area Light Sampling}
The basic algorithm is as follows:

\begin{alg}
\State trace primary camera ray
\ForAll{requested samples} 
	\State generate uniform sample on area light
	\State trace shadow ray from primary intersection to sample position
	\State integrate lighting
\EndFor
\end{alg}

The following components are involved:
\begin{itemize}
\item A ray tracer to find the closest scene intersection.
\item A ray tracer for the shadow rays (which finds any hit, not the closest one).
\item A ray generator for the primary rays.
\item A ray bouncer to compute samples starting from the primary intersections.
\end{itemize}

Our implementation will, for the moment, be set to use cuda, only.
RTA computes the different steps of a rendering algorithm in lock-step, 
	i.e.\ the steps outlined in the above algorithm are executed for each ray and only then is the next step considered.
Therefore, when generating samples on the light source, all primary intersections have been found.
This is a design decision to facilitate loose coupling of the different steps, 
	as well as to provide a simple mapping to cuda computations using multiple kernel calls.

\subsection{Obtaining a set of RTA Objects}
Getting the initial set of objects is done the easy way (and in the hope that it will be more portable in the long run)
	by using libhyb's plugin loading infrastructure, as follows.
Note that there is also material setup (\lstinline|gpu_materials|, lines~\ref{setup:gpumat1} and~\ref{setup:gpumat2}) mixed in, 
	which is (at the moment) implemented in the globillum program itself.

\begin{lstlisting}
static rta::cgls::connection *rta_connection = 0;
rta::basic_flat_triangle_list<rta::simple_triangle> *ftl = 0;
rta::cgls::connection::cuda_triangle_data *ctd = 0;
rta::cuda::material_t *gpu_materials = 0;@\label{setup:gpumat1}@

void setup_rta(std::string plugin) {
	vector<string> args;@\label{setup:plugin1}@
	plugin = "bbvh-cuda";

	if (plugin == "bbvh-cuda") {	// just in case
		args.push_back("-A");
		args.push_back("-b"); args.push_back("bsah");@\label{setup:bsah}@
		args.push_back("-t"); args.push_back("cis");
		args.push_back("-l"); args.push_back("2f4");
	}@\label{setup:plugin2}@

	typedef rta::cuda::simple_aabb box_t;
	typedef rta::cuda::simple_triangle tri_t;

	rta_connection = new rta::cgls::connection(plugin, args);@\label{setup:conn1}@
	ctd = rta::cgls::connection@\label{setup:conv1}@
	         ::convert_scene_to_cuda_triangle_data(the_scene);
	static rta::basic_flat_triangle_list<tri_t> the_ftl = ctd->cpu_ftl();@\label{setup:cpuftl1}@
	ftl = &the_ftl;
	rta::rt_set *set = 
		 new rta::rt_set(rta::plugin_create_rt_set(*ftl, rays_w, rays_h));@\label{setup:cpuftl2}@
	gpu_materials = rta::cuda::convert_and_upload_materials();@\label{setup:gpumat2}@

	if (!set->basic_ctor<box_t, tri_t>()->expects_host_triangles()) {@\label{setup:buildgpu1}@
		// the c'tor expects the triangle data on the gpu.
		set->as = set->basic_ctor<box_t,tri_t>()
		             ->build((tri_t::input_flat_triangle_list_t*)&ctd->ftl);@\label{setup:ctor}@
		set->basic_rt<box_t,tri_t>()
		    ->acceleration_structure(
				dynamic_cast<rta::basic_acceleration_structure<box_t,tri_t>*>
				(set->as));
	}@\label{setup:buildgpu2}@
}
\end{lstlisting}

Lines~\ref{setup:plugin1} to~\ref{setup:plugin2} specify the plugin to load and some parameters to it.
These can be found by calling RTA with the plugin you want to find information about:
\begin{lstlisting}[language=bash]
$ rta bbvh-cuda -- --help
\end{lstlisting}
The conversion of a \lstinline|cgls| scene to \lstinline|rta| triangle data is executed in line~\ref{setup:conv1}.
Following this conversion a host-side version of the triangle data is generated (line~\ref{setup:cpuftl1}) 
	and fed to the plugin (line~\ref{setup:cpuftl2}).
For cuda-only code this could have been done more economically by simply handing a single-element fake triangle list to the plugin.
The GPU BVH construction branch (lines~\ref{setup:buildgpu1} to~\ref{setup:buildgpu2}) will then build a proper BVH anyway.
In our use case, however, we want to be able to use BVHs other than LBVH, 
	i.e. a BVH build taking surface area heuristic into consideration (see line~\ref{setup:bsah}), 
	therefore we need to have host-side triangle data, too.

The \lstinline|rt_set| computed in line~\ref{setup:cpuftl2} contains the following items, managed by the called plugin to be compatible:
\begin{itemize}
\item the actual ray tracer (usually closest-hit),
\item the requested and build bvh (see comment on fake \lstinline|ftl| above),
\item a bvh contructor (used in line~\ref{setup:ctor} to trigger a gpu build), and
\item a default generator for primary rays.
\end{itemize}
Note that not ray bouncer is set, yet.

\subsection{Evaluate Material at Primary Intersection}
Material evaluation is implemented by a bouncer which does not generate further rays.
This is how a final computation after tracing can be integrated into RTA.
In the following code we derive from \lstinline|gpu_ray_bouncer| (line~\ref{matev:derv}) to get a \lstinline|bouncer| 
	that is properly integrated and supplied with GPU intersection data (see line~\ref{matev:gpudata}).
\begin{lstlisting}
template<typename _box_t, typename _tri_t>
		struct gpu_material_evaluator 
		: public cuda::gpu_ray_bouncer<forward_traits> {@\label{matev:derv}@
	declare_traits_types;
	cuda::material_t *materials;
	cuda::simple_triangle *tri_ptr;
	float3 *material_colors;
	cuda::cam_ray_generator_shirley *crgs;
	gpu_material_evaluator(uint w, uint h, cuda::material_t *materials, 
	                       cuda::simple_triangle *triangles, 
						   cuda::cam_ray_generator_shirley *crgs)
		: cuda::gpu_ray_bouncer<forward_traits>(w, h), 
		  materials(materials), material_colors(0), tri_ptr(triangles),
		  crgs(crgs) {
		checked_cuda(cudaMalloc(&material_colors, sizeof(float3)*w*h));
	}
	~gpu_material_evaluator() {
		checked_cuda(cudaFree(material_colors));
	}
	virtual bool trace_further_bounces() {
		return false;
	}
	virtual void evaluate_material() {
		rta::cuda
		   ::evaluate_material(this->w, this->h, 
						       this->gpu_last_intersection, @\label{matev:gpudata}@
						       tri_ptr, materials, material_colors, 
						       crgs->gpu_origin, crgs->gpu_direction);
	}
	virtual void bounce() {
		evaluate_material();
	}
	virtual std::string identification() {
		return "evaluate first-hit material on gpu.";
	}
};
\end{lstlisting}
The constructor takes the data dimensions which is required to allocate the intersection data and result colors.
Furthermore we have keep references to the materials, triangles and original rays to implement proper material lookup by
\begin{enumerate}
\item loading the triangle referenced by the intersection data,
\item looking up material (color and texture) information from the referenced triangles,
\item interpolating the triangle data to find texture coordinates and using ray differentials (using neighboring rays).
\end{enumerate}

Note how \lstinline|bounce| triggers an externally defined cuda kernel stub and \lstinline|trace_further_bounces| returns false to stop traversal.



\subsection{Using a Ray Bouncer to Implement the Sampling}
Our implementation of the area light sampler is an extension of the material evaluator presented above.
This is not required, but as we want to combine \emph{direct} illumation with already computed \emph{per-pixel} materials this plays out well.

Therefore, our sampling bouncer derives from the material evaluator and thus takes many of the same c'tor arguments.
Additionally we require a number of (rectangular) lights, a pool of uniform random data (to select light samples) and 
	the number of light samples to be computed.
Note that when a new position is chosen and rendering starts afresh the method \lstinline|new_pass| is called to reset the bouncer's data, 
	i.e. to enable it to later recognize that a call to \lstinline|bounce| is not a continuation of the previous computation.
\begin{lstlisting}
template<typename _box_t, typename _tri_t> 
		struct gpu_cgls_arealight_evaluator 
		: public gpu_material_evaluator<forward_traits> {
	declare_traits_types;
	cuda::cgls::rect_light *lights;
	int nr_of_lights;
	gi::cuda::halton_pool2f uniform_random_numbers;
	float3 *potential_sample_contribution,
	       *output_color; 
	int curr_bounce, samples;
	triangle_intersection<cuda::simple_triangle> *primary_intersection;

	gpu_cgls_arealight_evaluator(uint w, uint h,
	                             cuda::material_t *mtls, 
	                             cuda::simple_triangle *triangles, 
								 cuda::cam_ray_generator_shirley *crgs, 
								 cuda::cgls::rect_light *lights, 
								 int nr_of_lights,
								 gi::cuda::halton_pool2f rnd, 
								 int samples)
	: gpu_material_evaluator<forward_traits>(w, h, mtls, triangles, crgs), 
	  lights(lights), nr_of_lights(nr_of_lights), 
	  uniform_random_numbers(rnd), samples(samples) {
	    // allocate potential_sample_contribution
	    // allocate output_color
	    // allocate primary_intersection
		curr_bounce = 0;
	}
	~gpu_cgls_arealight_evaluator() {
		// free allocated stuff
	}
	virtual void new_pass() {
		curr_bounce = 0;
	}
	virtual bool trace_further_bounces() {
		return (curr_bounce < samples+1);
	}
	virtual void setup_new_arealight_sample() {
		rta::cuda::cgls
		   ::generate_rectlight_sample(
		   		this->w, this->h, lights, nr_of_lights, 
				this->crgs->gpu_origin, this->crgs->gpu_direction, 
				this->crgs->gpu_maxt,
				primary_intersection, this->tri_ptr, 
				uniform_random_numbers, 
				potential_sample_contribution, 
				curr_bounce);
	}
	virtual void integrate_light_sample() {
		rta::cuda::cgls
		   ::integrate_light_sample(
		   		this->w, this->h, this->gpu_last_intersection, 
				potential_sample_contribution,
				this->material_colors, output_color, curr_bounce-1);
	}
	virtual void bounce() {
		if (curr_bounce == 0) {
			// this has to be done before switching the intersection data.
			this->evaluate_material();
			triangle_intersection<cuda::simple_triangle> 
				*tmp = primary_intersection;
			primary_intersection = this->gpu_last_intersection;
			this->gpu_last_intersection = tmp;
			setup_new_arealight_sample();
		}
		else if (curr_bounce > 0) {
			integrate_light_sample();
			if (curr_bounce < samples)
				setup_new_arealight_sample();
		}
		++curr_bounce;
	}
};
\end{lstlisting}
When computing a bounce we have to take care to handle the first bounce differently from the others.
This is because the first bounce is after the computation of the primary intersections, and not part of the actual light sampling.
For this bounce we evaluate materials per pixel as described above and save the primary intersection data.
This is required as further bounces will overwrite the intersection data of the bouncer
	(upon which it is determined if the light sample is visible from the shading point).
We implement this efficiently by simply switching pointers.

As long there are samples left to compute we compute a new sample on the area light and store its contribution in 
	\lstinline|potential_sample_contribution|.
This is convenient as at the time we compute the sample all data required to determine its contribution (if the light is visible, hence `potential')
	is already available.
Sampling the light includes generating a new ray, 
	where the ray's \lstinline|max_t| parameter is used to only intersect geometry between the sample and the light.

Upon a completed light sample we integrate the lighting using a moving average to always have displayable data.
Whether the light source is visible can be read from the intersection data: if the intersection \lstinline|is| is empty (\lstinline|!is.valid()|)
	the light is unoccluded.

\subsection{Adding a Shadow Tracer for Light Sampling}
bla.

Note that switching the ray tracer from inside a bouncer requires using the progressive-framework 
	as the controlflow has to actually return from the first ray tracer.

\end{document}

% vim: set foldmethod=marker foldmarker=[[[,]]]: 


